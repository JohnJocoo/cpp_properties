% !TEX TS?program = pdflatexmk

\documentclass[12pt]{amsart}
\usepackage[utf8]{inputenc}
\usepackage{geometry} % see geometry.pdf on how to lay out the page. There's lots.
\usepackage{csquotes}
\usepackage[english]{babel}
\usepackage[
backend=biber,
style=authortitle,
sorting=ynt
]{biblatex}
\usepackage{listings}

\geometry{a4paper} % or letter or a5paper or ... etc
% \geometry{landscape} % rotated page geometry

\addbibresource{property_manifest.bib}

% See the ``Article customise'' template for come common customisations

\title{C++ Properties - a library solution}
\author{Anton Frolov}
%\date{} % delete this line to display the current date

%%% BEGIN DOCUMENT
\begin{document}

\maketitle

\clearpage\tableofcontents

\clearpage\section{Introduction}
	\textit{Property} is a useful concept that has a native implementation in dozen of programming languages that support \textit{OOP}. Itself \textit{properties} have originated as part of \textit{object-oriented paradigm} to represent the \textit{public} parts of object's state and as well to provide unified interface to directly modify state if allowed. The ability to access \textit{property} as ordinary \textit{data field} is the key feature \textit{properties} are praised for. More practical definition of \textit{property} exist:
\begin{displayquote}	
	A property, in some object-oriented programming languages, is a special sort of class member, intermediate in functionality between a field (or data member) and a method. The syntax for reading and writing of properties is like for fields, but property reads and writes are (usually) translated to 'getter' and 'setter' method calls. \cite{wiki_properties}
\end{displayquote}
	A code example is appropriate here:
\begin{lstlisting}[language=C++] % use C++ highlighting for C# 

class Zoo 
{
    private int monkeys; // private field
    
    public int Monkeys // public property 
    {  
        get
        {
            return this.monkeys;
        }
        set 
        {
            if (value > 0) 
                this.monkeys = value;
        }
    } // public int Monkeys
} // class Zoo 

Zoo zoo = new Zoo();
int monkeys_tmp = zoo.Monkeys;
zoo.Monkeys = 17;
zoo.Monkeys += 1; // clearer than "zoo.setMonkeys(zoo.getMonkeys() + 1)"

\end{lstlisting}
	Sadly, \textit{property} remains to be an alien concept for C++ and its community, despite following:
\begin{enumerate}
\item Large amount of proposals being pushed to C++ committee regarding \textit{properties}. \footnote{At the time of writing search "properties" on https://isocpp.org/forums/iso-c-standard-future-proposals gives about 300 results. Sure not all of papers found are about properties as a language feature, but still there are a lot of interest in standardising properties.}
\item At least two popular compilers implement \textit{properties} as an extension. \footnote{Visual C++ https://msdn.microsoft.com/en-us/library/yhfk0thd.aspx and clang https://clang.llvm.org/docs/LanguageExtensions.html.}
\item People keep reinventing them as a library solutions. \footnote{Qt http://doc.qt.io/qt-5/properties.html and C++ CX https://docs.microsoft.com/en-us/cpp/cppcx/properties-c-cx are libraries with properties implemented.}
\end{enumerate}
\bigskip	Such narrow-mindedness often leads to either importability \footnote{Often public fields are used, such poor design leads to inability to change class implementation without changing user code.} or inconvenience and code bloat:
\begin{lstlisting}[language=C++] 

class Zoo 
{
    int monkeys; // private field
    
public:
    int getMonkeys() const // public getter 
    {
        return monkeys;
    }  

    void setMonkeys( int value ) // public setter 
    {
        if (value > 0)
            monkeys = value;
    }  
} // class Zoo 

Zoo zoo;
zoo.setMonkeys(zoo.getMonkeys() + 1); // I want "zoo.Monkeys++"

\end{lstlisting}
	After struggling with get/set functions my whole carrier, I stepped across an inefficient and bad-designed property library. As an attempt to improve code quality of my employer I've created yet another library that implement \textit{properties}. This paper aims to justify library design and convince potential users to give it a try.

\clearpage\section{Problem}
\subsection{Encapsulation and get/set}
\subsection{Public field phobia}

\printbibliography

\end{document}

